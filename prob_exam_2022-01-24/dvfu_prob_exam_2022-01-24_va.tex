% arara: xelatex
\documentclass[12pt]{article}

\usepackage{tikz} % картинки в tikz
\usepackage{microtype} % свешивание пунктуации

\usepackage{array} % для столбцов фиксированной ширины

\usepackage{indentfirst} % отступ в первом параграфе

\usepackage{sectsty} % для центрирования названий частей
\allsectionsfont{\centering}

\usepackage{amsmath} % куча стандартных математических плюшек

\usepackage{comment}
\usepackage{amsfonts}

\usepackage[top=2cm, left=1cm, right=1cm, bottom=2cm]{geometry} % размер текста на странице

\usepackage{lastpage} % чтобы узнать номер последней страницы

\usepackage{enumitem} % дополнительные плюшки для списков
%  например \begin{enumerate}[resume] позволяет продолжить нумерацию в новом списке
\usepackage{caption}

\usepackage{hyperref} % гиперссылки

\usepackage{multicol} % текст в несколько столбцов


\usepackage{fancyhdr} % весёлые колонтитулы
\pagestyle{fancy}
\lhead{ДВФУ, теория вероятностей}
\chead{}
\rhead{Экзамен, 2022-01-24}
\lfoot{Вариант B.1.1.529}
\cfoot{Да пребудет с тобой Сила!}
\rfoot{}
\renewcommand{\headrulewidth}{0.4pt}
\renewcommand{\footrulewidth}{0.4pt}



\usepackage{todonotes} % для вставки в документ заметок о том, что осталось сделать
% \todo{Здесь надо коэффициенты исправить}
% \missingfigure{Здесь будет Последний день Помпеи}
% \listoftodos --- печатает все поставленные \todo'шки


% более красивые таблицы
\usepackage{booktabs}
% заповеди из докупентации:
% 1. Не используйте вертикальные линни
% 2. Не используйте двойные линии
% 3. Единицы измерения - в шапку таблицы
% 4. Не сокращайте .1 вместо 0.1
% 5. Повторяющееся значение повторяйте, а не говорите "то же"


\usepackage{fontspec}
\usepackage{polyglossia}

\setmainlanguage{russian}
\setotherlanguages{english}

% download "Linux Libertine" fonts:
% http://www.linuxlibertine.org/index.php?id=91&L=1
\setmainfont{Linux Libertine O} % or Helvetica, Arial, Cambria
% why do we need \newfontfamily:
% http://tex.stackexchange.com/questions/91507/
\newfontfamily{\cyrillicfonttt}{Linux Libertine O}

\AddEnumerateCounter{\asbuk}{\russian@alph}{щ} % для списков с русскими буквами
\setlist[enumerate, 2]{label=\asbuk*),ref=\asbuk*}

%% эконометрические сокращения
\DeclareMathOperator{\Cov}{Cov}
\DeclareMathOperator{\Corr}{Corr}
\DeclareMathOperator{\Var}{Var}
\DeclareMathOperator{\E}{E}
\def \hb{\hat{\beta}}
\def \hs{\hat{\sigma}}
\def \htheta{\hat{\theta}}
\def \s{\sigma}
\def \hy{\hat{y}}
\def \hY{\hat{Y}}
\def \v1{\vec{1}}
\def \e{\varepsilon}
\def \he{\hat{\e}}
\def \z{z}
\def \hVar{\widehat{\Var}}
\def \hCorr{\widehat{\Corr}}
\def \hCov{\widehat{\Cov}}
\def \cN{\mathcal{N}}
\def \P{\mathbb{P}}


\begin{document}


\begin{enumerate}
  \item Два рыболова независимо друг от друга выловили по рыбе. 
  Рыбины в реке всегда встречаются равновероятно длиной в один, два, три или четыре метра.  

  \begin{enumerate}
    \item Найдите вероятность того, что суммарная длина выловленных рыбин больше пяти метров.
    \item Найдите вероятность того, что первый выловил трехметровую рыбину, если суммарная длина выловленных рыбин больше пяти метров.
  \end{enumerate}

  \item Вечный студент Петр решает одну задачу каждый день. Решения в разные дни независимы. 
  Петр становится всё умнее с каждым днём, потому на $n$-ый день вероятность ошибочного решения равна $1/2^n$. 
  

  Найдите ожидание и дисперсию количества ошибочно решённых задач за бесконечный период его студенчества. 

  \item Сумма на вкладе Александра Ивановича Корейко неслучайно растёт по формуле $S(t) = 100\exp(0.1t)$, где $t$ измеряется в годах. 
  Из-за охоты Остапа Бендера Корейко вынужден закрыть свой вклад в случайный момент времени $X$. 
  Величина $X$ распределена экcпоненциально с ожиданием в четыре года.

  \begin{enumerate}
    \item Найдите ожидаемую сумму вклада на момент закрытия. 
    \item Найдите функцию плотности вклада на момент закрытия. 
  \end{enumerate}

  \item Джеймс Бонд десантируется в случайную точку внутри треугольника с вершинами $(-1, 0)$, $(1, 0)$
  и $(0, 1)$. Обозначим $X$ и $Y$ — координаты приземления Джеймса Бонда. 

  \begin{enumerate}
    \item Найдите $\Cov(X, Y)$ и $\E(Y \mid X)$.
    \item Зависимы ли величины $X$ и $Y$?
  \end{enumerate}

  \item Каждый день Илон Маск стоит в пробке экспоненциальное время $X_i$, в среднем 5 минут в день. 
  Ожидания за разные дни независимы. 

  \begin{enumerate}
    \item Какова вероятность того, что за 100 дней он простоит в пробках больше 10 часов в сумме?
    \item Илон Маск хочет сделать заявление, что прождал в пробках больше 10 часов. 
    На какой день ему нужно запланировать заявление, чтобы оно оказалось верным с вероятностью $0.99$?
  \end{enumerate}
  

  При записи ответа можно использовать функцию распределения $\Phi()$ стандартной нормальной случайной величины
  и обратную к ней. 

 % \item Джон и Иван играют в азартную игру: они извлекают карты из хорошо перемешанной колоды 
 % в 52 карты в случайном порядке. 

%  Джон выигрывает, если за первым извлеченным королём будет идти король пик. 
%  Иван выигрывает, если за первым извлеченным королём будет идти туз пик. 
%  Если за первым королём идёт не король пик, и не туз пик, то огра оканчивается в ничью. 


%  \begin{enumerate}
%    \item В чью пользу эта игра? Объясните, почему. 
%    \item Какова вероятность ничьей?
%  \end{enumerate}
  

  \item Начинающий секретарь Васисуалий при печати распоряжения шефа случайно 
  равновероятно нажимает заглавные русские буквы, включая букву «Ё».
  
  Сколько всего в среднем нажатий потребуется Васисуалию, чтобы напечатать «НУ»?
\end{enumerate}

\end{document}
