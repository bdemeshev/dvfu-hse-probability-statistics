
\begin{question}
Исследователи Машенька и Вовочка, не зная друг о друге, каждый день по
всем правилам статистики строят 95\%-й доверительные интервалы для
математического ожидания \(\mu\).

Выборка у них общая на двоих, и каждый день --- новая. При этом Машенька
знает истинную дисперсию, а Вовочка --- нет. Все наблюдения одинаково
нормально распределены и независимы.

Выберите верное утверждение.
\begin{answerlist}
  \item Машенькины интервалы всегда уже Вовочкиных
  \item нет верного ответа
  \item Машенькины интервалы всегда шире Вовочкиных
  \item Машенькины интервалы всегда правее Вовочкиных
  \item Машенькины интервалы бывают как шире, так и уже Вовочкиных
  \item Машенькины интервалы всегда левее Вовочкиных
\end{answerlist}
\end{question}


