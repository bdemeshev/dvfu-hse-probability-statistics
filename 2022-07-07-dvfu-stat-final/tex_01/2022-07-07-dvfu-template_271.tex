% arara: xelatex: {shell: yes}
%% arara: biber
%% arara: xelatex: {shell: yes}
%% arara: xelatex: {shell: yes}
\documentclass[12pt]{article}

\usepackage{etoolbox} % для условия if-else
\newtoggle{excerpt} % помечаем, что это отрывок, а далее в тексте может использовать
\toggletrue{excerpt}
% команду \iftoggle{excerpt}{yes}{no}

\usepackage{tikz} % картинки в tikz
\usepackage{microtype} % свешивание пунктуации

\usepackage{array} % для столбцов фиксированной ширины

\usepackage{indentfirst} % отступ в первом параграфе

\usepackage{sectsty} % для центрирования названий частей
\allsectionsfont{\centering}

\usepackage{amsmath} % куча стандартных математических плюшек

\usepackage{comment}
\usepackage{amsfonts}

\usepackage[top=2cm, left=1.2cm, right=1.2cm, bottom=2cm]{geometry} % размер текста на странице

\usepackage{lastpage} % чтобы узнать номер последней страницы

\usepackage{enumitem} % дополнительные плюшки для списков
%  например \begin{enumerate}[resume] позволяет продолжить нумерацию в новом списке
\usepackage{caption}


\usepackage{fancyhdr} % весёлые колонтитулы
\pagestyle{fancy}
\lhead{Теория вероятностей и математическая статистика — ДВФУ}
\chead{}
\rhead{Финальный экзамен, 2022-07-07}
\lfoot{}
\cfoot{}
\rfoot{\thepage/\pageref{LastPage}}
\renewcommand{\headrulewidth}{0.4pt}
\renewcommand{\footrulewidth}{0.4pt}



\usepackage{todonotes} % для вставки в документ заметок о том, что осталось сделать
% \todo{Здесь надо коэффициенты исправить}
% \missingfigure{Здесь будет Последний день Помпеи}
% \listoftodos --- печатает все поставленные \todo'шки


% более красивые таблицы
\usepackage{booktabs}
% заповеди из докупентации:
% 1. Не используйте вертикальные линни
% 2. Не используйте двойные линии
% 3. Единицы измерения - в шапку таблицы
% 4. Не сокращайте .1 вместо 0.1
% 5. Повторяющееся значение повторяйте, а не говорите "то же"

\usepackage{multicol}


\usepackage{fontspec}
\usepackage{polyglossia}

\setmainlanguage{russian}
\setotherlanguages{english}

% download "Linux Libertine" fonts:
% http://www.linuxlibertine.org/index.php?id=91&L=1
\setmainfont{Linux Libertine O} % or Helvetica, Arial, Cambria
% why do we need \newfontfamily:
% http://tex.stackexchange.com/questions/91507/
\newfontfamily{\cyrillicfonttt}{Linux Libertine O}

\AddEnumerateCounter{\asbuk}{\russian@alph}{щ} % для списков с русскими буквами
\setlist[enumerate, 2]{label=\alph*),ref=\alph*} % \alph* \asbuk* \arabic*

%% эконометрические сокращения
\DeclareMathOperator{\Cov}{Cov}
\DeclareMathOperator{\Corr}{Corr}
\DeclareMathOperator{\Var}{Var}
\DeclareMathOperator{\E}{E}
\newcommand \hb{\hat{\beta}}
\newcommand \hs{\hat{\sigma}}
\newcommand \htheta{\hat{\theta}}
\newcommand \s{\sigma}
\newcommand \hy{\hat{y}}
\newcommand \hY{\hat{Y}}
\newcommand \e{\varepsilon}
\newcommand \he{\hat{\e}}
\newcommand \z{z}
\newcommand \hVar{\widehat{\Var}}
\newcommand \hCorr{\widehat{\Corr}}
\newcommand \hCov{\widehat{\Cov}}
\newcommand \cN{\mathcal{N}}
\let\P\relax
\DeclareMathOperator{\P}{\mathbb{P}}

\DeclareMathOperator{\plim}{plim}



\usepackage{color,url,amsthm,amssymb,longtable,eurosym}
\newenvironment{question}{\item }{}
% \newenvironment{solution}{}{}
\excludecomment{solution}
\newenvironment{answerlist}{\begin{multicols}{3}\begin{enumerate}}{\end{enumerate}\end{multicols}}
\newcommand{\setzerocols}{
  \renewenvironment{answerlist}{\begin{enumerate}}{\end{enumerate}}
}
\newcommand{\setncols}[1]{
  \renewenvironment{answerlist}{\begin{multicols}{#1}\begin{enumerate}}{\end{enumerate}\end{multicols}}
}


\newcommand{\answerbox}{\raisebox{3mm}{%
    \fbox{%
          \begin{minipage}[t]{2mm}%
              \hspace*{2mm}%
              \vspace*{0.3cm}
          \end{minipage}
         }
    }
}

\newcommand{\answerline}{
    \answerbox\hspace*{-1mm}a \hspace*{3mm}
    \answerbox\hspace*{-1mm}b \hspace*{3mm}
    \answerbox\hspace*{-1mm}c \hspace*{3mm}
    \answerbox\hspace*{-1mm}d \hspace*{3mm}
    \answerbox\hspace*{-1mm}e \hspace*{3mm}
    \answerbox\hspace*{-1mm}f \hspace*{3mm}
}

\newcommand{\fiobox}{
\fbox{
  \begin{minipage}{42em}
    Имя, фамилия и номер группы:\vspace*{3ex}\par
    \noindent\dotfill\vspace{2mm}
  \end{minipage}
}
}

% делаем короче интервал в списках
\setlength{\itemsep}{0pt}
\setlength{\parskip}{0pt}
\setlength{\parsep}{0pt}

\begin{document}
\fiobox

\begin{multicols}{2}
\begin{enumerate}
    \item \answerline
    \item \answerline
    \item \answerline
    \item \answerline
    \item \answerline
    
    \item \answerline
    \item \answerline
    \item \answerline
    \item \answerline
    \item \answerline
    
    \item \answerline
    \item \answerline
    \item \answerline
    \item \answerline
    \item \answerline
    
    \item \answerline
    \item \answerline
    \item \answerline
    \item \answerline
    \item \answerline

    \item \answerline
    \item \answerline
    \item \answerline
    \item \answerline
    \item \answerline
    
    \item \answerline
    \item \answerline
    
\end{enumerate}
\end{multicols}

\newpage
Удачи!
\newpage

\fiobox

\begin{enumerate}
\setncols{2}

\begin{question}
Величины \(X\) и \(Y\) одинаково распределены с нулевым математическим
ожиданием и дисперсией \(7\). Вектор \((X, Y)\) имеет многомерное
нормальное распределение с корреляцией \(0.4\).

Найдите \(\mathbb{E}(Y\mid X = 2)\).
\begin{answerlist}
  \item \(0.8\)
  \item нет верного ответа
  \item \(0\)
  \item \(0.84\)
  \item \(0.4\)
  \item \(5.88\)
\end{answerlist}
\end{question}



\setncols{3}

\begin{question}
Пусть \(X_1\), \ldots, \(X_n\) --- выборка объема \(n\) из некоторого
распределения с конечным математическим ожиданием.

Выберите несмещенную и состоятельную оценку математического ожидания.
\begin{answerlist}
  \item \(\frac{X_1}{2 n}+\frac{X_2+\ldots+X_{n-2}}{n-1}+\frac{X_n}{2 n}\)
  \item \(\frac{X_1}{2 n}+\frac{X_2+\ldots+X_{n-2}}{n-2}+\frac{X_n}{2 n}\)
  \item нет верного ответа
  \item \(\frac{1}{3} X_1 + \frac{2}{3} X_2\)
  \item \(\frac{X_1}{2 n}+\frac{X_2+\ldots+X_{n-1}}{n-2}-\frac{X_n}{2 n}\)
  \item \(\frac{X_1+X_2}{2}\)
\end{answerlist}
\end{question}

\begin{solution}
========
\end{solution}



\begin{question}
Оценка \(\hat a_n\) неизвестного параметра \(a\) асимптотически
нормальная и несмещённая. По выборке из 200 наблюдений оказалось, что
\(\hat a_n = 3\) с оценкой дисперсии
\(\widehat{\mathrm{Var}}(\hat a_n) = 3\).

Найдите правую границу симметричного двустороннего 95\%-го
доверительного интервала для параметра \(a\).
\begin{answerlist}
  \item 9.85
  \item 13.32
  \item нет верного ответа
  \item 16.78
  \item 15.05
  \item 6.39
\end{answerlist}
\end{question}




\begin{question}
Длины катетов в сантиметрах прямоугольного треугольника являются
модулями независимых стандартных нормальных случайных величин.

Какую пороговую длину гипотенуза этого треугольника превышает с
вероятностью \(0.05\)?
\begin{answerlist}
  \item нет верного ответа
  \item \(0.21\)
  \item \(4.61\)
  \item \(5.99\)
  \item \(0.1\)
  \item \(0.68\)
\end{answerlist}
\end{question}

\begin{solution}
========
\end{solution}


\setncols{2}

\begin{question}
Величины \(X_i\) независимы и одинаково распределены с математическим
ожиданием \(\mathbb{E}(X_i) = 2 a + 9\).

По выборке из 500 наблюдений оказалось, что \(\bar X = 15\).

Найдите оценку \(\hat a\) методом моментов.
\begin{answerlist}
  \item 3.5
  \item 4
  \item 3
  \item нет верного ответа
  \item 4.5
  \item 2.5
\end{answerlist}
\end{question}



\setzerocols

\begin{question}
Выберите верное утверждение о связи уровня значимости \(\alpha\) и
\(P\)-значения.
\begin{answerlist}
  \item \(P\)-значение монотонно падает с ростом \(\alpha\)
  \item \(P\)-значение монотонно растёт с ростом \(\alpha\)
  \item нет верного ответа
  \item \(\alpha\) и \(P\)-значение не связаны
  \item \(P\)-значение случайно, ожидание от него монотонно падает с ростом
\(\alpha\)
  \item \(P\)-значение случайно, ожидание от него монотонно растёт с ростом
\(\alpha\)
\end{answerlist}
\end{question}



\setncols{3}

\begin{question}
Проверяется гипотеза \(H_0\): \(\theta = \gamma\) против альтернативной
гипотезы \(H_a\): \(\theta \neq \gamma\), где \(\theta\) и \(\gamma\)
--- два неизвестных параметра.

Выберите верное утверждение о распределении статистики отношения
правдоподобия, \(LR\).
\begin{answerlist}
  \item если верна \(H_a\), то \(LR \sim \chi_2^2\)
  \item и при \(H_0\), и при \(H_a\), \(LR \sim \chi_1^2\)
  \item если верна \(H_a\), то \(LR \sim \chi_1^2\)
  \item если верна \(H_0\), то \(LR \sim \chi_1^2\)
  \item нет верного ответа
  \item и при \(H_0\), и при \(H_a\), \(LR \sim \chi_2^2\)
\end{answerlist}
\end{question}

\begin{solution}
========
\end{solution}



\begin{question}
Величины \(X\) и \(Y\) одинаково распределены с нулевым математическим
ожиданием и дисперсией \(5\). Вектор \((X, Y)\) имеет многомерное
нормальное распределение с корреляцией \(0.5\).

Найдите \(\mathrm{Var}(Y\mid X = 3)\).
\begin{answerlist}
  \item нет верного ответа
  \item \(0.75\)
  \item \(5\)
  \item \(3.75\)
  \item \(0.5\)
  \item \(1.5\)
\end{answerlist}
\end{question}




\begin{question}
Величины \(X_i\) независимы и равномерны на отрезке \([-a;2a]\).

Оцените \(a\) методом максимального правдоподобия по выборке из трех
наблюдений: -5, -3, 13.
\begin{answerlist}
  \item 6
  \item 5.5
  \item 7.5
  \item нет верного ответа
  \item 7
  \item 6.5
\end{answerlist}
\end{question}




\begin{question}
Величина \(X\) имеет \(F\)-распределение с 3 и 12 степенями свободы.

Какое распределение имеет величина \(Y = X^{-1}\)?
\begin{answerlist}
  \item \(F_{1/3, 1/12}\)
  \item \(\chi^2_{15}\)
  \item \(F_{1/12, 1/3}\)
  \item \(F_{12, 3}\)
  \item нет верного ответа
  \item \(F_{3, 12}\)
\end{answerlist}
\end{question}




\begin{question}
Рассмотрим хи-квадрат случайную величину с \(n\) степенями свободы.
Укажите множество всех возможных значений, принимаемых данной случайной
величиной с ненулевой вероятностью:
\begin{answerlist}
  \item \(\{0, 1, \ldots, n\}\)
  \item \([0,n^2]\)
  \item \((0, \infty)\)
  \item нет верного ответа
  \item \([0,n]\)
  \item \(\{x\in R:\sum_{i=1}^{n}{x_{{}}^{2}}=1\}\)
\end{answerlist}
\end{question}

\begin{solution}
========
\end{solution}


\setncols{2}

\begin{question}
Известно, что \(\mathbb{E}(\hat a) = 0.7a + 3\), функция правдоподобия
регулярна и информация Фишера равна \(I_F(a) = 1/a^2\).

Найдите теоретическую нижнюю границу \(\mathrm{Var}(\hat a)\).
\begin{answerlist}
  \item \(0.7 a^2\)
  \item \(0.49 a^2\)
  \item \(a^2\)
  \item нет верного ответа
  \item \(3 a^2\)
  \item \(9 a^2\)
\end{answerlist}
\end{question}



\setncols{3}

\begin{question}
Отличница Машенька получает только 8, 9 или 10. За все годы обучения
Маша получила 60 восьмёрок, 30 девяток и 40 десяток.

Найдите значение статистики Пирсона для проверки гипотезы о том, все
отличные оценки имеют равную вероятность.
\begin{answerlist}
  \item 40.77
  \item 70.77
  \item нет верного ответа
  \item 10.77
  \item 60.77
  \item 50.77
\end{answerlist}
\end{question}




\begin{question}
Геродот Геликарнасский проверяет гипотезу \(H_0: \; \mu=2\). Лог-функция
правдоподобия имеет вид
\(\ell(\mu,\nu)=-\frac{n}{2}\ln (2\pi)-\frac{n}{2}\ln \nu -\frac{\sum_{i=1}^n(x_i-\mu)^2}{2\nu}\).

Найдите оценка максимального правдоподобия для \(\nu\) при
предположении, что \(H_0\) верна.
\begin{answerlist}
  \item \(\frac{\sum x_i^2 - 4\sum x_i}{n}\)
  \item нет верного ответа
  \item \(\frac{\sum x_i^2 - 4\sum x_i}{n}+2\)
  \item \(\frac{\sum x_i^2 - 4\sum x_i+2}{n}\)
  \item \(\frac{\sum x_i^2 - 4\sum x_i+4}{n}\)
  \item \(\frac{\sum x_i^2 - 4\sum x_i}{n}+4\)
\end{answerlist}
\end{question}

\begin{solution}
========
\end{solution}



\begin{question}
Геродот Геликарнасский проверяет гипотезу \(H_0: \; \mu=2\). Лог-функция
правдоподобия имеет вид
\(\ell(\mu,\nu)=-\frac{n}{2}\ln (2\pi)-\frac{n}{2}\ln \nu -\frac{\sum_{i=1}^n(x_i-\mu)^2}{2\nu}\).

Найдите оценка максимального правдоподобия для \(\nu\) при
предположении, что \(H_0\) верна.
\begin{answerlist}
  \item \(\frac{\sum x_i^2 - 4\sum x_i}{n}+4\)
  \item нет верного ответа
  \item \(\frac{\sum x_i^2 - 4\sum x_i+2}{n}\)
  \item \(\frac{\sum x_i^2 - 4\sum x_i+4}{n}\)
  \item \(\frac{\sum x_i^2 - 4\sum x_i}{n}+2\)
  \item \(\frac{\sum x_i^2 - 4\sum x_i}{n}\)
\end{answerlist}
\end{question}

\begin{solution}
========
\end{solution}



\begin{question}
Кот Матроскин поймал 20 рыб. Совсем маленьких, весом до 1 кг, он
отпустил. Оставшиеся три рыбы весили 2 кг, 3 кг и 4 кг.

Найдите значение выборочной функции распределения массы пойманных рыб в
точке 3.5 кг.
\begin{answerlist}
  \item 0.8
  \item 0.95
  \item 0.85
  \item нет верного ответа
  \item 0.9
  \item 0.75
\end{answerlist}
\end{question}



\setzerocols

\begin{question}
Известно, что величины \(X_1\), \ldots, \(X_{300}\) независимы и имеют
экспоненциальное распределение с интенсивностью \(\lambda\), и
\(\ln L(\lambda)\) --- логарифмическая функция правдоподобия.

Найдите
\(\mathbb{E}\left(\frac{\partial \ln L(\lambda)}{\partial\lambda} \right)\).
\begin{answerlist}
  \item нет верного ответа
  \item \(300\lambda\)
  \item \(0\)
  \item \(300/\lambda\)
  \item \(300\lambda^2\)
  \item \(300/\lambda^2\)
\end{answerlist}
\end{question}



\setncols{3}

\begin{question}
Каждое утро в 8:00 Иван Андреевич Крылов, либо завтракает, либо уже
позавтракал. В это же время кухарка либо заглядывает к Крылову, либо
нет.

По таблице сопряженности вычислите статистику \(\chi^2\) Пирсона для
тестирования гипотезы о том, что визиты кухарки не зависят от того,
позавтракал ли уже Крылов или нет.

\begin{longtable}[]{@{}lll@{}}
\toprule
& Кухарка заходит & Кухарка не заходит \\
\midrule
\endhead
Крылов завтракает & 200 & 40 \\
Крылов уже позавтракал & 25 & 100 \\
\bottomrule
\end{longtable}
\begin{answerlist}
  \item 79
  \item 179
  \item 39
  \item 100
  \item нет верного ответа
  \item 139
\end{answerlist}
\end{question}




\begin{question}
По 100 наблюдениям получена оценка метода максимального правдоподобия,
\(\hat\theta = 20\), также известны значения лог-функции правдоподобия
\(\ell(20) = -10\) и \(\ell(0)= - 50\).

С помощью критерия отношения правдоподобия, \(LR\), проверьте гипотезу
\(H_0\): \(\theta = 0\) против \(H_0\): \(\theta \neq 0\) на уровне
значимости 5\%.
\begin{answerlist}
  \item \(LR = 40\), \(H_0\) не отвергается
  \item нет верного ответа
  \item \(LR = 80\), \(H_0\) отвергается
  \item Критерий неприменим
  \item \(LR = 60\), \(H_0\) не отвергается
  \item \(LR = 40\), \(H_0\) отвергается
\end{answerlist}
\end{question}

\begin{solution}
========
\end{solution}



\begin{question}
Пусть \(X_1\), \ldots, \(X_n\) --- выборка объема \(n\) из некоторого
распределения с конечным математическим ожиданием.

Выберите несмещенную и состоятельную оценку математического ожидания.
\begin{answerlist}
  \item \(\frac{X_1}{2 n}+\frac{X_2+\ldots+X_{n-1}}{n-2}-\frac{X_n}{2 n}\)
  \item \(\frac{X_1+X_2}{2}\)
  \item \(\frac{X_1}{2 n}+\frac{X_2+\ldots+X_{n-2}}{n-1}+\frac{X_n}{2 n}\)
  \item \(\frac{1}{3} X_1 + \frac{2}{3} X_2\)
  \item \(\frac{X_1}{2 n}+\frac{X_2+\ldots+X_{n-2}}{n-2}+\frac{X_n}{2 n}\)
  \item нет верного ответа
\end{answerlist}
\end{question}

\begin{solution}
========
\end{solution}


\setzerocols

\begin{question}
Величина \(X\) имеет \(t\)-распределение с 5 степенями свободы.

Какое распределение имеет величина \(Y = X^2\)?
\begin{answerlist}
  \item \(F_{1, 5}\)
  \item \(F_{5, 5}\)
  \item \(F_{5, 1}\)
  \item \(t_{25}\)
  \item нет верного ответа
  \item \(\chi^2_{5}\)
\end{answerlist}
\end{question}



\setncols{3}

\begin{question}
Величины \(X_i\) независимы и распределены по Пуассону с параметром
интенсивности \(\lambda\).

Выберите несмещённую оценку для \(\mathbb{E}(X_i)\).
\begin{answerlist}
  \item \(\sum_{i=1}^n X_i / (n - 1)\)
  \item \(\sum_{i=1}^n X_i / (n + 1)\)
  \item \(\sum_{i=1}^n X_i^2 / n\)
  \item \(\sum_{i=1}^n (X_i - \bar X)^2 / (n - 1)\)
  \item нет верного ответа
  \item \(\sum_{i=1}^n X_i^2 / (n - 1)\)
\end{answerlist}
\end{question}




\begin{question}
Известно, что величины \(X_1\), \ldots, \(X_{200}\) независимы и имеют
экспоненциальное распределение с интенсивностью \(\lambda\), и
\(\ln L(\lambda)\) --- логарифмическая функция правдоподобия.

Найдите
\(\mathbb{E}\left(\frac{\partial \ln L(\lambda)}{\partial\lambda} \right)\).
\begin{answerlist}
  \item \(200/\lambda\)
  \item \(200\lambda\)
  \item нет верного ответа
  \item \(200/\lambda^2\)
  \item \(200\lambda^2\)
  \item \(0\)
\end{answerlist}
\end{question}




\begin{question}
Величины \(X_i\) независимы и распределены по Пуассону с параметром
интенсивности \(\lambda\).

Выберите несмещённую оценку для \(\mathbb{E}(X_i)\).
\begin{answerlist}
  \item \(\sum_{i=1}^n X_i^2 / n\)
  \item нет верного ответа
  \item \(\sum_{i=1}^n X_i^2 / (n - 1)\)
  \item \(\sum_{i=1}^n X_i / (n + 1)\)
  \item \(\sum_{i=1}^n X_i / (n - 1)\)
  \item \(\sum_{i=1}^n (X_i - \bar X)^2 / (n - 1)\)
\end{answerlist}
\end{question}




\begin{question}
Рассмотрим хи-квадрат случайную величину с \(n\) степенями свободы.
Укажите множество всех возможных значений, принимаемых данной случайной
величиной с ненулевой вероятностью:
\begin{answerlist}
  \item \(\{x\in R:\sum_{i=1}^{n}{x_{{}}^{2}}=1\}\)
  \item \([0,n^2]\)
  \item нет верного ответа
  \item \([0,n]\)
  \item \(\{0, 1, \ldots, n\}\)
  \item \((0, \infty)\)
\end{answerlist}
\end{question}

\begin{solution}
========
\end{solution}



\begin{question}
Теоретическая информация Фишера о параметре \(a\) описывается функцией
\(I_F(a)\), при этом \(a = 4b\).

Какой функцией описывается теоретическая информация Фишера о параметре
\(b\)?
\begin{answerlist}
  \item \(I_F(4 b)\)
  \item нет верного ответа
  \item \(16 I_F(4 b)\)
  \item \(4 I_F(b/4)\)
  \item \(4 I_F(4 b)\)
  \item \(4 I_F(b)\)
\end{answerlist}
\end{question}




\begin{question}
Величины \(X_i\) независимы и одинаково распределены с математическим
ожиданием \(\mathbb{E}(X_i) = 2 a + 1\).

По выборке из 500 наблюдений оказалось, что \(\bar X = 18\).

Найдите оценку \(\hat a\) методом моментов.
\begin{answerlist}
  \item 9.5
  \item 10
  \item нет верного ответа
  \item 9
  \item 8
  \item 8.5
\end{answerlist}
\end{question}



\end{enumerate}
\end{document}
