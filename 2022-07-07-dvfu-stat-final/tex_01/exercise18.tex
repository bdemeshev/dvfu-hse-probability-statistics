
\begin{question}
Каждое утро в 8:00 Иван Андреевич Крылов, либо завтракает, либо уже
позавтракал. В это же время кухарка либо заглядывает к Крылову, либо
нет.

По таблице сопряженности вычислите статистику \(\chi^2\) Пирсона для
тестирования гипотезы о том, что визиты кухарки не зависят от того,
позавтракал ли уже Крылов или нет.

\begin{longtable}[]{@{}lll@{}}
\toprule
& Кухарка заходит & Кухарка не заходит \\
\midrule
\endhead
Крылов завтракает & 200 & 40 \\
Крылов уже позавтракал & 25 & 100 \\
\bottomrule
\end{longtable}
\begin{answerlist}
  \item 79
  \item 179
  \item 39
  \item 100
  \item нет верного ответа
  \item 139
\end{answerlist}
\end{question}


