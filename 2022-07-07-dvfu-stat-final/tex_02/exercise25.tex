
\begin{question}
Исследователь Вовочка при проверке гипотезы о равенстве математического
ожидания константе по ошибке вместо \(t\)-распределения использует
стандартное нормальное.

Как изменяются при этом вероятность ошибки первого рода \(\alpha\) и
ошибки второго рода \(\beta\)?
\begin{answerlist}
  \item \(\alpha\) падает, \(\beta\) изменяется непредсказуемо
  \item \(\alpha\) растёт, \(\beta\) растёт
  \item нет верного ответа
  \item \(\alpha\) падает, \(\beta\) падает
  \item \(\alpha\) растёт, \(\beta\) падает
  \item \(\alpha\) падает, \(\beta\) растёт
\end{answerlist}
\end{question}


