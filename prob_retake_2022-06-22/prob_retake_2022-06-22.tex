% arara: xelatex
\documentclass[12pt]{article}

\usepackage{tikz} % картинки в tikz
\usepackage{microtype} % свешивание пунктуации

\usepackage{array} % для столбцов фиксированной ширины

\usepackage{indentfirst} % отступ в первом параграфе

\usepackage{sectsty} % для центрирования названий частей
\allsectionsfont{\centering}

\usepackage{amsmath} % куча стандартных математических плюшек

\usepackage{comment}
\usepackage{amsfonts}

\usepackage{verbatim}


\usepackage[colorlinks=true, linkcolor=blue]{hyperref}

\usepackage[top=2cm, left=1cm, right=1cm, bottom=2cm]{geometry} % размер текста на странице

\usepackage{lastpage} % чтобы узнать номер последней страницы

\usepackage{enumitem} % дополнительные плюшки для списков
%  например \begin{enumerate}[resume] позволяет продолжить нумерацию в новом списке
\usepackage{caption}

\usepackage{hyperref} % гиперссылки

\usepackage{multicol} % текст в несколько столбцов


\usepackage{fancyhdr} % весёлые колонтитулы
\pagestyle{fancy}
\lhead{ДВФУ-теория вероятностей}
\chead{}
\rhead{пересдача}
\lfoot{2022-06-22}
\cfoot{}
\rfoot{}
\renewcommand{\headrulewidth}{0.4pt}
\renewcommand{\footrulewidth}{0.4pt}



\usepackage{todonotes} % для вставки в документ заметок о том, что осталось сделать
% \todo{Здесь надо коэффициенты исправить}
% \missingfigure{Здесь будет Последний день Помпеи}
% \listoftodos --- печатает все поставленные \todo'шки


% более красивые таблицы
\usepackage{booktabs}
% заповеди из докупентации:
% 1. Не используйте вертикальные линни
% 2. Не используйте двойные линии
% 3. Единицы измерения - в шапку таблицы
% 4. Не сокращайте .1 вместо 0.1
% 5. Повторяющееся значение повторяйте, а не говорите "то же"


\usepackage{fontspec}
\usepackage{polyglossia}

\setmainlanguage{russian}
\setotherlanguages{english}

% download "Linux Libertine" fonts:
% http://www.linuxlibertine.org/index.php?id=91&L=1
\setmainfont{Linux Libertine O} % or Helvetica, Arial, Cambria
% why do we need \newfontfamily:
% http://tex.stackexchange.com/questions/91507/
\newfontfamily{\cyrillicfonttt}{Linux Libertine O}

\AddEnumerateCounter{\asbuk}{\russian@alph}{щ} % для списков с русскими буквами
\setlist[enumerate, 2]{label=\asbuk*),ref=\asbuk*}

%% эконометрические сокращения
\DeclareMathOperator{\Cov}{Cov}
\DeclareMathOperator{\Corr}{Corr}
\DeclareMathOperator{\Var}{Var}
\DeclareMathOperator{\E}{E}
\def \hb{\hat{\beta}}
\def \hs{\hat{\sigma}}
\def \htheta{\hat{\theta}}
\def \s{\sigma}
\def \hy{\hat{y}}
\def \hY{\hat{Y}}
\def \v1{\vec{1}}
\def \e{\varepsilon}
\def \he{\hat{\e}}
\def \z{z}
\def \hVar{\widehat{\Var}}
\def \hCorr{\widehat{\Corr}}
\def \hCov{\widehat{\Cov}}
\def \cN{\mathcal{N}}
\def \P{\mathbb{P}}
\def \id {\mathrm{id}\_\mathrm{for}\_\mathrm{online}}

\begin{document}

Во всех задачах, где требуется найти вероятности, связанные с нормальным распределением, можно оставлять ответ со стандартной нормальной функцией распределения. 

\begin{enumerate}

\item Компания проводит AB-тестирование, чтобы определить наилучший дизайн сайта. 
Каждый пользователь с вероятностью 40\% попадает в A-группу и видит старый дизайн сайта, 
а с вероятностью 60\% попадает в B-группу и видит новый дизайн сайта. 
Среди A-группы 70\% процентов одобряют дизайн, среди B-группы дизайн одобряют 80\%.

\begin{enumerate}
  \item Какова вероятность того, что случайно выбранный пользователь окажется в B-группе и одобрит при этом дизайн?
  \item Известно, что Вася одобряет дизайн сайта. Какова вероятность того, что он из A-группы?
\end{enumerate}

  \item Дана совместная функция плотности пары величин $X$, $Y$:
  \[
    f(x, y) = \begin{cases} 
      x+ y, \text{ при } x\in [0;1], y\in [0;1]; \\
      0, \text{ иначе.}
    \end{cases}
  \]

Найдите $\E(Y^2)$, $\P(X > 0.5 \mid Y > 0.5)$.


\item Срок службы холодильника имеет экспоненциальное распределение. В среднем один холодильник
бесперебойно работает 5 лет. Завод предоставляет гарантию 3 лет на свои изделия. 
Примерно 80\% потребителей аккуратно хранят все бумаги, необходимые, чтобы воспользоваться
гарантией.

\begin{enumerate}
  \item Какой процент потребителей в среднем обращается за гарантийным ремонтом?
  \item Какова вероятность того, что из 1000 потребителей за гарантийным ремонтом обратится более 35\%
покупателей?
\end{enumerate}

\item Случайный вектор $(X,Y)^T$ имеет двумерное нормальное распределение
с математическим ожиданием $(0,0)^T$ и ковариационной матрицей
\[
C = \begin{pmatrix}
9 & -1 \\
-1 & 4 \\
\end{pmatrix}.
\]

Найдите вероятности $\P(2X+3Y>1)$ и $\P(2X+3Y>1 \mid X=0)$. 

\item Каждую весну дед Мазай плавая на лодке спасает в среднем 9 зайцев, дисперсия
количества спасённых зайцев за одну весну равна 12. Точный закон распределения числа зайцев
неизвестен.

В каких пределах лежит вероятность того, что за одну весну дед Мазай
спасёт более 11 зайцев?

\item Известно, что последовательность величин $R_n$ сходится по вероятности к константе $1/2$. 
Например, можно представить себе, что $R_n$ — доля орлов при $n$ бросках монетки. 

Найдите предел по вероятности последовательности $Q_n$, где 
\[
Q_n = \frac{2(R_n - 0.5)^2}{R_n \ln (2R_n) + (1-R_n)\ln (2 (1-R_n))}.
\]



\end{enumerate}

\end{document}
